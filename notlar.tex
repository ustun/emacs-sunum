% Created 2014-12-02 Tue 16:53
\documentclass[11pt]{article}
\usepackage[utf8]{inputenc}
\usepackage[T1]{fontenc}
\usepackage{fixltx2e}
\usepackage{graphicx}
\usepackage{longtable}
\usepackage{float}
\usepackage{wrapfig}
\usepackage{rotating}
\usepackage[normalem]{ulem}
\usepackage{amsmath}
\usepackage{textcomp}
\usepackage{marvosym}
\usepackage{wasysym}
\usepackage{amssymb}
\usepackage{hyperref}
\tolerance=1000
\author{Ustun Ozgur}
\date{\today}
\title{Emacs'e Giris}
\hypersetup{
  pdfkeywords={},
  pdfsubject={},
  pdfcreator={Emacs 24.4.1 (Org mode 8.2.10)}}
\begin{document}

\maketitle
\tableofcontents


\section{Tarihce}
\label{sec-1}
\begin{itemize}
\item The extensible, customizable, self-documenting, real-time display editor
\item 1976 yilinda Richard Stallman ve Guy Steele
\item Editor MACros
\item Vi ile birlikte en eski iki editor
\end{itemize}

\section{Temel Kullanim}
\label{sec-2}

\subsection{Temel Kavramlar}
\label{sec-2-1}

Frame - pencere
Window - pencere icindeki bolme
Buffer - tampon. gecici ya da kalici metin. acik dosya ya da metin ciktisi. (ornegin kod ciktisi)
Echo area: En alttaki alan
Mini-buffer: Echo area'da kullanici input'u beklendigi anda gosterilen alan

\subsection{Ctrl ve Alt tuslari ile tus kombinasyonlari}
\label{sec-2-2}

Ctrl-x ya da C-x
Alt-x (Cmd-x) ya da M-x (Meta tusu)
C-x C-g: Once Ctrl-x sonra Ctrl-g
C-u 5 C-k: Once C-u sonra 5 sonra C-k gibi

\subsection{Cursor hareketleri}
\label{sec-2-3}

\begin{itemize}
\item f forward (ileri)
\item b backward (geri)

\item C-f Ctrl-f : Bir karakter ileri
\item C-b Ctrl-b : Bir karakter geri
\item M-f: Bir kelime ileri
\item M-b: Bir kelime geri

\item C-v: Page down
\item M-v: Page up

\item M->: En sona git
\item M-<: En basa git

\item C-s: Search forward
\item C-r: Search backward

\item C-g: Cancel

\item C-d: Delete char
\item M-d: Delete word
\item M-Backspace: Backward delete word
\end{itemize}

\subsection{Cut-Copy-Paste}
\label{sec-2-4}

Emacs'in bu konudaki kavramlari da biraz farklidir.

\begin{itemize}
\item Cut   -> Kill (C-w)
\item Copy  -> Copy (M-w)
\item Paste -> Yank (C-y)
\end{itemize}

Secim yapmak icin: Ctrl-Space daha sonra istenilen yere kadar git ve
C-w. Daha sonra istenen yere git ve C-y ile yapistir.

Satir sonuna kadar secim yapmadan kesmek icin: C-k

Kesilen her sey kill-ring denen yere gider, clipboard manager gibi dusunulebilir.


\subsection{Basit dosya acma/kapama}
\label{sec-2-5}

\begin{itemize}
\item C-x C-f: Find file

\item C-x C-b: List Buffers. Acik dosyalari goruntule

\item C-x k: Kill buffer. Buffer'i kapat.

\item C-x C-s: Save buffer

\item C-x d: Dired mode. Klasoru goruntuler.
\end{itemize}

\subsection{Pencere yonetimi}
\label{sec-2-6}

\begin{itemize}
\item C-x 2: Yatay olarak ikiye bol (su anki window'u iki window'a bol)

\item C-x 3: Dikey olarak ikiye bol

\item C-x 1: Diger window'lari kapat

\item C-x 0: Mevcut window'u kapat

\item C-x o: Other Window. Diger window'a gecis yap.
\end{itemize}

\subsection{Modlar}
\label{sec-2-7}

\begin{itemize}
\item Temel modlar: Genelde dillere bagli modlar (kisayollar, syntax
highlighting)

\item Yardimci modlar: Birden fazla dilde yararli olabilecek fonksiyonlar
(ornegin show-paren-mode eslesen parantezi gosterir.)
\end{itemize}

\subsection{Demo}
\label{sec-2-8}

\section{Emacs'in Farki}
\label{sec-3}

Buraya kadar gorduklerimiz herhangi bir editorde yapabilecegimiz
seyler. Bu kisimda emacs'in neden diger editorlerden daha guclu
oldugunu gorecegiz. Emacs'in en onemli yani extensible olmasidir, yani
kullanici, emacs'i istedigi gibi programlayarak kendine gore
ozellestirebilir. Bunun sebebi emacs'in emacs lisp dili ile
gelistirebilmesidir. Emacs'le birlikte gelen bircok ozellik de bu dil
ile yazilmistir. (performans gerektiren kisimlar c dili ile
yazilmistir.

\subsection{Emacs Lisp Ozeti}
\label{sec-3-1}

\begin{itemize}
\item Lisp: list processing
\item Neredeyse tek syntax liste
\item Ic ice listeler tanimlanabilir
\item (eylem arg1 arg2 \ldots{} argN)
\item Genel olarak: (fonksiyon-adi arg1 arg2 \ldots{} argN)
\end{itemize}

\subsection{Emacs Lisp Ornekleri}
\label{sec-3-2}

\begin{itemize}
\item (+ 1 2) => 3
\item (+ (* 3 4) (* 5 6)) => 42
\item (set 'isim "Ustun") => set
\item (setq isim "Ustun") => set quoted
\item (defun topla (a b)  (+ a b))
\item (topla 3 4) => 7
\item (defun carp  (a b)  (* a b))
\item (carp 5 6) => 30
\item (funcall (lambda (a b) (* a b)) 4 5) => Lambda (isimsiz (anonim) fonksiyon)
\item (message "Hello World")
\item (message "Hello World \%s" "GNUBilkent")
\end{itemize}

\subsection{Fonksiyonlar ve Komutlar}
\label{sec-3-3}

\begin{itemize}
\item Aslinda her tus bir komuta (command) baglidir. Ornegin C-x C-s
\end{itemize}
tuslarina bastigimizda save-buffer isimli komut calistirilir.

\begin{itemize}
\item Bir tusun hangi komuta bagli oldugunu C-h k (describe-key) ile
\end{itemize}
gorebiliriz.

\begin{itemize}
\item Ornegin C-h k daha sonra C-x C-s tuslarina bastigimizda save-buffer
\end{itemize}
komutu calistirilir.

\begin{itemize}
\item Tum komutlara M-x ile ulasabiliriz.

\item M-x find-file

\item M-x save-buffer
\end{itemize}

\subsection{Yeni Komut Eklemek}
\label{sec-3-4}

\begin{itemize}
\item Yeni komutlar ve fonksiyonlar ekleyebiliriz.

\item Ornegin, bir satir geri gidip bos satir ekleyen bir komut yazalim.

\item Ingredients:
\begin{itemize}
\item Bir satir ileri gitme: (forward-line N)
\item Bir satir geri gitme: (forward-line -1)
\item Bos satir ekleme: (newline)
\end{itemize}

\item Bu ikisini kullanarak yeni bir fonksiyon yazalim:
\end{itemize}

(defun bos-satir-ekle ()
    (forward-line -1)
    (newline))

\begin{itemize}
\item Bu haliyle bir fonksiyon yarattik, ancak bu bir komut degil. Komut
olmasi icin (interactive) cagrisini eklememiz gerek.

(defun bos-satir-ekle ()
(interactive)
(forward-line -1)
(newline))

\item Simdi M-x yaptigimizda bos-satir-ekle komutunu calistirabiliriz.
\end{itemize}

\subsection{Komutumuza Kisayol Eklemek}
\label{sec-3-5}

\begin{itemize}
\item Temelde iki turlu kisayol eklenebilir: Global ve aktif moda (dosya turune gore).

\item (global-set-key (kbd "C-o") 'bos-satir-ekle)

\item Artik Ctrl-o ile komutumuzu calistirabiliriz.

\item Konumumuzu korumak icin her seyi save-excursion (seyahati kaydet) ile wrap edelim.
\end{itemize}
\begin{verbatim}
  (defun bos-satir-ekle ()
  (interactive)
(save-excursion
(forward-line -1)
  (newline)))
\end{verbatim}

\subsection{Degisiklikleri Kaydetmek}
\label{sec-3-6}
\begin{itemize}
\item Home klasorunde .emacs.d klasoru icinde init.el (eskiden direk home icinde .emacs kullanilirdi)
\item `C-x C-e` ile son fonksiyonu yeniden tanimlayabiliriz.
\item `C-M-x` ile fonksiyon sonuna gitmeden yeniden tanimlayabiliriz.
\end{itemize}

\subsection{Yardim Komutlarir}
\label{sec-3-7}

\begin{enumerate}
\item describe-function
\item describe-key
\item describe-mode
\item describe-k $\gamma$
\item apropos: arama

\item C-h t: Tutorial
\item C-h r: Manual
\item C-h i: Info (Emacs'in man pageleri gibi. Manual'a buradan da erisilebilir

\item find-function
\item C-h S (info-lookup)
\item find-library
\item elisp-index-search
\end{enumerate}

\section{Eklenti (Mod) Yukleme}
\label{sec-4}

\subsection{Paket yoneticisi}
\label{sec-4-1}

Emacs 24'te paket yoneticisi entegre gelmekte, ancak ana depo daha az
paket icerdigi icin ekstra bir depo adresi eklemeli.
\begin{verbatim}
(require 'package)
(add-to-list 'package-archives '("melpa" . "http://melpa.org/packages/") t)
(package-initialize)
\end{verbatim}

Daha sonra `M-x package-list-packages`

Paketleri I (install) ile sec, daha sonra X (execute) ile yukle.

\url{http://melpa.org} adresinden tum paketler incelenebilir.

\subsection{En Populer Paketler}
\label{sec-4-2}



\begin{itemize}
\item ido mode
\begin{itemize}
\item kolay dosya acmak icin
\item flexible matching icin: (setq ido-enable-flex-matching t)
\item ido-ubiquitous
\item ido-vertical-mode
\end{itemize}

\item tema secimi: (load-theme 'leuven) ya da M-x load-theme

\item font secimi: M-x customize-face ENTER default ENTER
\begin{itemize}
\item daha sonra save for future sessions
\end{itemize}

\item windmove: Shift+yon tuslari ile pencereler arasi gecis

\item smex: M-x icin ido benzeri ozellik.

\item helm: ido+smex alternatifi, daha guclu.

\item recentf-mode

\item projectile-mode

\item magit: git icin

\item dired

\item org-mode: Not tutma, rapor hazirlama, planlama yapma

\item flycheck: Aninda syntax checking.

\item autocomplete-mode: Autocomplete mode

\item imenu ve idomenu

\item yasnippet: Snippetler

\item M-x list-packages

\item Butun bunlarin otomatik olarak kuruldugu bir depo icin emacs-prelude
kullanilabilir.
\end{itemize}

\subsection{Diger ilginc paketler}
\label{sec-4-3}

\begin{itemize}
\item eww: Web Browser

\item gnus: Mail reader

\item erc: Chat reader

\item occur

\item jedi

\item ag

\item emmet

\item deft

\item hl-line-mode

\item visual-line-mode

\item helm

\item fringe-mode

\item customize-variable

\item describe-variable

\item eldoc-mode
\end{itemize}

\subsection{Extra}
\label{sec-4-4}

x^2 + 1 = \sum_{i=0}{x}


\begin{center}
\begin{tabular}{lr}
x & y\\
\hline
3 & 4\\
ssss & 333\\
 & \\
\end{tabular}
\end{center}
% Emacs 24.4.1 (Org mode 8.2.10)
\end{document}
